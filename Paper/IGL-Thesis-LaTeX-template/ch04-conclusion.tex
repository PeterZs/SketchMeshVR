\chapter{Conclusion and Outlook}


\section{Future work}
The developed software offers a lot of potential for future expansion and improvements. The software misses some of the useful tools that are present in other modelling software such as simple predefined shapes (like cubes or cylinders), mirroring and merging meshes. Using a second VR controller (e.g. the Oculus Touch controller) gives enough buttons to map these functions to, allowing us to implement them without the addition of menu, thus staying with the principle of "intuitive" hand gestures. Adding an extra controller also allows for the user to switch between smooth and sharp curve deformation. The functionality for this is embedded in the software, but is not mapped to any button because all buttons of the first controller are already occupied by other functions. 

Another functionality that would be useful in many cases is the the possibility to use blueprint images. Letting the user define multiple images, taken from different angles, of the object they want to model, allows them to trace these different silhouettes and precisely recreate the desired object. \todo{adapt this sentence to something readable} 

Additionally, the quality of the created meshes could greatly be improved by applying intermediate remeshing. As can be seen from the mesh examples, the triangle sizes differ greatly between triangles that are part of the initial created shape and triangles that are part of a subsequent cut surface or extruded part. This makes the appearance of the mesh rather chaotic and this could be avoided by remeshing every time a new surface is created. The purpose of the remeshing would be to make the edge lengths more uniform across the different parts of the mesh, resulting in a much more homogeneous mesh structure.

Final possibilities for improvements to the software lie in improving the interface. Implementing hand avatars for hand orientation instead of a simple sphere greatly increase the feeling of immersion. Additionally displaying the currently selected tool modes (for example cutting versus extrusion) and enabling the user to navigate the mesh with hand gestures or the controllers are valuable additions that will improve the usability. 

\section{Conclusion}