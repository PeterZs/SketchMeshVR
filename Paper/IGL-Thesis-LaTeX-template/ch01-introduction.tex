% set counter to n-1:
\setcounter{chapter}{0}

\chapter{Introduction}

Achtung! Dieser Blindtext wird gerade durch 130 Millionen Rezeptoren Ihrer Netzhaut erfasst. Die Zellen werden dadurch in einen Erregungszustand versetzt, der sich �ber den Sehnerv in dem hinteren Teil Ihres Gehirns ausbreitet. Von dort aus �bertr�gt sich die Erregung in Sekundenbruchteilen auch in andere Bereiche Ihres Grosshirns. Ihr Stirnlappen wird stimuliert. Von dort aus gehen jetzt Willensimpulse aus, die Ihr zentrales Nervensystem in konkrete Handlungen umsetzt. Kopf und Augen reagieren bereits. Sie folgen dem Text, nehmen die darin enthaltenen Informationen auf wie ein Schwamm. Nicht auszudenken, was mit Ihnen h�tte passieren k�nnen, wenn dieser Blindtext durch einen echten Text ersetzt worden w�re. Achtung! Dieser Blindtext wird gerade durch 130 Millionen Rezeptoren Ihrer Netzhaut erfasst. Die Zellen werden dadurch in einen Erregungszustand versetzt, der sich �ber den Sehnerv in dem hinteren Teil Ihres Gehirns ausbreitet. Von dort aus �bertr�gt sich die Erregung in Sekundenbruchteilen auch in andere Bereiche Ihres Grosshirns. Ihr Stirnlappen wird stimuliert. Von dort aus gehen jetzt Willensimpulse aus, die Ihr zentrales Nervensystem in konkrete Handlungen umsetzt. Kopf und Augen reagieren bereits. Sie folgen dem Text, nehmen die darin enthaltenen Informationen auf wie ein Schwamm. Nicht auszudenken, was mit Ihnen h�tte passieren k�nnen, wenn dieser Blindtext durch einen echten Text ersetzt worden w�re. Achtung! Dieser Blindtext wird gerade durch 130 Millionen Rezeptoren Ihrer Netzhaut erfasst. Die Zellen werden dadurch in einen Erregungszustand versetzt, der sich �ber den Sehnerv in dem hinteren Teil Ihres Gehirns ausbreitet. Von dort aus �bertr�gt sich die Erregung in Sekundenbruchteilen auch in andere Bereiche Ihres Grosshirns. Ihr Stirnlappen wird stimuliert. Von dort aus gehen jetzt Willensimpulse aus, die Ihr zentrales Nervensystem in konkrete Handlungen umsetzt. Kopf und Augen reagieren bereits. Sie folgen dem Text, nehmen die darin enthaltenen Informationen auf wie ein Schwamm. Nicht auszudenken, was mit Ihnen h�tte passieren k�nnen, wenn dieser Blindtext durch einen echten Text ersetzt worden w�re. Achtung! Dieser Blindtext wird gerade durch 130 Millionen Rezeptoren Ihrer Netzhaut erfasst. Die Zellen werden dadurch in einen Erregungszustand versetzt, der sich �ber den Sehnerv in dem hinteren Teil Ihres Gehirns ausbreitet. Von dort aus �bertr�gt sich die Erregung in Sekundenbruchteilen auch in andere Bereiche Ihres Grosshirns.

Ihr Stirnlappen wird stimuliert. Von dort aus gehen jetzt Willensimpulse aus, die Ihr zentrales Nervensystem in konkrete Handlungen umsetzt. Kopf und Augen reagieren bereits. Sie folgen dem Text, nehmen die darin enthaltenen Informationen auf wie ein Schwamm. Nicht auszudenken, was mit Ihnen h�tte passieren k�nnen, wenn dieser Blindtext durch einen echten Text ersetzt worden w�re. Achtung! Dieser Blindtext wird gerade durch 130 Millionen Rezeptoren Ihrer Netzhaut erfasst. Die Zellen werden dadurch in einen Erregungszustand versetzt, der sich �ber den Sehnerv in dem hinteren Teil Ihres Gehirns ausbreitet. Von dort aus �bertr�gt sich die Erregung in Sekundenbruchteilen auch in andere Bereiche Ihres Grosshirns. Ihr Stirnlappen wird stimuliert. Von dort aus gehen jetzt Willensimpulse aus, die Ihr zentrales Nervensystem in konkrete Handlungen umsetzt. Kopf und Augen reagieren bereits. Sie folgen dem Text, nehmen die darin enthaltenen Informationen auf wie ein Schwamm. Nicht auszudenken, was mit Ihnen h�tte passieren k�nnen, wenn dieser Blindtext durch einen echten Text ersetzt worden w�re. Achtung! Dieser Blindtext wird gerade durch 130 Millionen Rezeptoren Ihrer Netzhaut erfasst. Die Zellen werden dadurch in einen Erregungszustand versetzt, der sich �ber den Sehnerv in dem hinteren Teil Ihres Gehirns ausbreitet. Von dort aus �bertr�gt sich die Erregung in Sekundenbruchteilen auch in andere Bereiche Ihres Grosshirns. Ihr Stirnlappen wird stimuliert. Von dort aus gehen jetzt Willensimpulse aus, die Ihr zentrales Nervensystem in konkrete Handlungen umsetzt.
