% set counter to n-1:
\setcounter{chapter}{0}

\chapter{Introduction}
\label{chap:intro}
Digital 3D shape modeling is a field that has received a lot of attention over the past years as a result of increasing rendering possibilities and a growing demand for 3-dimensional digital content due to the digitalization of movie-making, fabrication, engineering, architecture, product design, medicine and many other disciplines. The recent surge in interest for VR (Virtual Reality) has driven the development of better GPUs (Graphics Processing Unit) even more, while also further increasing the demand for digital 3D content. 
The process of 3D shape modeling has so far been almost exclusively suitable for trained professionals or users with a lot of experience. While 3D modeling software generally comes with an abundance of options and possibilities, it also usually has a steep learning curve. This causes 3D modeling to become inaccessible to novice users who would like to create (simple) 3D models, but who do not have the time to learn how to use these programs. 

Sketch-based 3D modeling mostly seeks to simplify the process of 3D modeling in order to make it more accessible for this user group, but also to provide professional users with a means of quickly making rough drafts or simple base models to which details can be added by other software, or to experiment. Its goal is to provide the user with intuitive ways to interact with and edit the mesh that is being modelled. Previous work regarding sketch-based modeling tools has produced multiple software products which present the user with this intuitive method of 3D modeling. However, these have only been made available for use with a PC and mouse or on mobile~\cite{GravitySketch}, and not for usage in VR. 

Recently a variety of art creation tools for usage in VR have been created. These applications range from painting in 3D to voxel modeling and 3D sculpting. VR gives artists the possibility to look at what they are creating from a literally new perspective. Directly modeling in 3D can give an improved perception of the proportions of different parts of a model and the relations between them.

The goal of this thesis is to develop a sketch-based 3D modeling program explicitly created for usage in a VR setting. This combines the simplicity and intuitiveness of sketch-based designing with the immersiveness of modeling in actual 3D space and the possibility of viewing the results immediately in 3D. The software is targeted to users that are new to 3D modeling and should therefore be straightforward to use and easy and quick to learn.

Chapter \ref{chap:related} discusses previous work that is done in the fields of sketch-based 3D modeling and give several examples of software for 3D modeling in virtual reality. 

In Chapter \ref{chap:system}, we give a detailed description of the VR sketch-based 3D modeling program that was developed as part of this thesis, and discuss how it differs from previous work. It describes the algorithms that were used and gives a description of the user interface including the rationale behind the design decisions that were made. 

Next, in Chapter \ref{chap:results} we let target users test our software and summarize the feedback that they have given. We also compare the VR sketch-based 3D modeling tool with a version of the software that is developed for non-VR use with a computer and mouse. 

Finally, Chapter \ref{chap:conclusion} provides ideas for future improvements on the software and summarizes the findings of this thesis.